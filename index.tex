% Options for packages loaded elsewhere
% Options for packages loaded elsewhere
\PassOptionsToPackage{unicode}{hyperref}
\PassOptionsToPackage{hyphens}{url}
\PassOptionsToPackage{dvipsnames,svgnames,x11names}{xcolor}
%
\documentclass[
  letterpaper,
  DIV=11,
  numbers=noendperiod]{scrreprt}
\usepackage{xcolor}
\usepackage{amsmath,amssymb}
\setcounter{secnumdepth}{2}
\usepackage{iftex}
\ifPDFTeX
  \usepackage[T1]{fontenc}
  \usepackage[utf8]{inputenc}
  \usepackage{textcomp} % provide euro and other symbols
\else % if luatex or xetex
  \usepackage{unicode-math} % this also loads fontspec
  \defaultfontfeatures{Scale=MatchLowercase}
  \defaultfontfeatures[\rmfamily]{Ligatures=TeX,Scale=1}
\fi
\usepackage{lmodern}
\ifPDFTeX\else
  % xetex/luatex font selection
\fi
% Use upquote if available, for straight quotes in verbatim environments
\IfFileExists{upquote.sty}{\usepackage{upquote}}{}
\IfFileExists{microtype.sty}{% use microtype if available
  \usepackage[]{microtype}
  \UseMicrotypeSet[protrusion]{basicmath} % disable protrusion for tt fonts
}{}
\makeatletter
\@ifundefined{KOMAClassName}{% if non-KOMA class
  \IfFileExists{parskip.sty}{%
    \usepackage{parskip}
  }{% else
    \setlength{\parindent}{0pt}
    \setlength{\parskip}{6pt plus 2pt minus 1pt}}
}{% if KOMA class
  \KOMAoptions{parskip=half}}
\makeatother
% Make \paragraph and \subparagraph free-standing
\makeatletter
\ifx\paragraph\undefined\else
  \let\oldparagraph\paragraph
  \renewcommand{\paragraph}{
    \@ifstar
      \xxxParagraphStar
      \xxxParagraphNoStar
  }
  \newcommand{\xxxParagraphStar}[1]{\oldparagraph*{#1}\mbox{}}
  \newcommand{\xxxParagraphNoStar}[1]{\oldparagraph{#1}\mbox{}}
\fi
\ifx\subparagraph\undefined\else
  \let\oldsubparagraph\subparagraph
  \renewcommand{\subparagraph}{
    \@ifstar
      \xxxSubParagraphStar
      \xxxSubParagraphNoStar
  }
  \newcommand{\xxxSubParagraphStar}[1]{\oldsubparagraph*{#1}\mbox{}}
  \newcommand{\xxxSubParagraphNoStar}[1]{\oldsubparagraph{#1}\mbox{}}
\fi
\makeatother

\usepackage{color}
\usepackage{fancyvrb}
\newcommand{\VerbBar}{|}
\newcommand{\VERB}{\Verb[commandchars=\\\{\}]}
\DefineVerbatimEnvironment{Highlighting}{Verbatim}{commandchars=\\\{\}}
% Add ',fontsize=\small' for more characters per line
\usepackage{framed}
\definecolor{shadecolor}{RGB}{241,243,245}
\newenvironment{Shaded}{\begin{snugshade}}{\end{snugshade}}
\newcommand{\AlertTok}[1]{\textcolor[rgb]{0.68,0.00,0.00}{#1}}
\newcommand{\AnnotationTok}[1]{\textcolor[rgb]{0.37,0.37,0.37}{#1}}
\newcommand{\AttributeTok}[1]{\textcolor[rgb]{0.40,0.45,0.13}{#1}}
\newcommand{\BaseNTok}[1]{\textcolor[rgb]{0.68,0.00,0.00}{#1}}
\newcommand{\BuiltInTok}[1]{\textcolor[rgb]{0.00,0.23,0.31}{#1}}
\newcommand{\CharTok}[1]{\textcolor[rgb]{0.13,0.47,0.30}{#1}}
\newcommand{\CommentTok}[1]{\textcolor[rgb]{0.37,0.37,0.37}{#1}}
\newcommand{\CommentVarTok}[1]{\textcolor[rgb]{0.37,0.37,0.37}{\textit{#1}}}
\newcommand{\ConstantTok}[1]{\textcolor[rgb]{0.56,0.35,0.01}{#1}}
\newcommand{\ControlFlowTok}[1]{\textcolor[rgb]{0.00,0.23,0.31}{\textbf{#1}}}
\newcommand{\DataTypeTok}[1]{\textcolor[rgb]{0.68,0.00,0.00}{#1}}
\newcommand{\DecValTok}[1]{\textcolor[rgb]{0.68,0.00,0.00}{#1}}
\newcommand{\DocumentationTok}[1]{\textcolor[rgb]{0.37,0.37,0.37}{\textit{#1}}}
\newcommand{\ErrorTok}[1]{\textcolor[rgb]{0.68,0.00,0.00}{#1}}
\newcommand{\ExtensionTok}[1]{\textcolor[rgb]{0.00,0.23,0.31}{#1}}
\newcommand{\FloatTok}[1]{\textcolor[rgb]{0.68,0.00,0.00}{#1}}
\newcommand{\FunctionTok}[1]{\textcolor[rgb]{0.28,0.35,0.67}{#1}}
\newcommand{\ImportTok}[1]{\textcolor[rgb]{0.00,0.46,0.62}{#1}}
\newcommand{\InformationTok}[1]{\textcolor[rgb]{0.37,0.37,0.37}{#1}}
\newcommand{\KeywordTok}[1]{\textcolor[rgb]{0.00,0.23,0.31}{\textbf{#1}}}
\newcommand{\NormalTok}[1]{\textcolor[rgb]{0.00,0.23,0.31}{#1}}
\newcommand{\OperatorTok}[1]{\textcolor[rgb]{0.37,0.37,0.37}{#1}}
\newcommand{\OtherTok}[1]{\textcolor[rgb]{0.00,0.23,0.31}{#1}}
\newcommand{\PreprocessorTok}[1]{\textcolor[rgb]{0.68,0.00,0.00}{#1}}
\newcommand{\RegionMarkerTok}[1]{\textcolor[rgb]{0.00,0.23,0.31}{#1}}
\newcommand{\SpecialCharTok}[1]{\textcolor[rgb]{0.37,0.37,0.37}{#1}}
\newcommand{\SpecialStringTok}[1]{\textcolor[rgb]{0.13,0.47,0.30}{#1}}
\newcommand{\StringTok}[1]{\textcolor[rgb]{0.13,0.47,0.30}{#1}}
\newcommand{\VariableTok}[1]{\textcolor[rgb]{0.07,0.07,0.07}{#1}}
\newcommand{\VerbatimStringTok}[1]{\textcolor[rgb]{0.13,0.47,0.30}{#1}}
\newcommand{\WarningTok}[1]{\textcolor[rgb]{0.37,0.37,0.37}{\textit{#1}}}

\usepackage{longtable,booktabs,array}
\usepackage{calc} % for calculating minipage widths
% Correct order of tables after \paragraph or \subparagraph
\usepackage{etoolbox}
\makeatletter
\patchcmd\longtable{\par}{\if@noskipsec\mbox{}\fi\par}{}{}
\makeatother
% Allow footnotes in longtable head/foot
\IfFileExists{footnotehyper.sty}{\usepackage{footnotehyper}}{\usepackage{footnote}}
\makesavenoteenv{longtable}
\usepackage{graphicx}
\makeatletter
\newsavebox\pandoc@box
\newcommand*\pandocbounded[1]{% scales image to fit in text height/width
  \sbox\pandoc@box{#1}%
  \Gscale@div\@tempa{\textheight}{\dimexpr\ht\pandoc@box+\dp\pandoc@box\relax}%
  \Gscale@div\@tempb{\linewidth}{\wd\pandoc@box}%
  \ifdim\@tempb\p@<\@tempa\p@\let\@tempa\@tempb\fi% select the smaller of both
  \ifdim\@tempa\p@<\p@\scalebox{\@tempa}{\usebox\pandoc@box}%
  \else\usebox{\pandoc@box}%
  \fi%
}
% Set default figure placement to htbp
\def\fps@figure{htbp}
\makeatother


% definitions for citeproc citations
\NewDocumentCommand\citeproctext{}{}
\NewDocumentCommand\citeproc{mm}{%
  \begingroup\def\citeproctext{#2}\cite{#1}\endgroup}
\makeatletter
 % allow citations to break across lines
 \let\@cite@ofmt\@firstofone
 % avoid brackets around text for \cite:
 \def\@biblabel#1{}
 \def\@cite#1#2{{#1\if@tempswa , #2\fi}}
\makeatother
\newlength{\cslhangindent}
\setlength{\cslhangindent}{1.5em}
\newlength{\csllabelwidth}
\setlength{\csllabelwidth}{3em}
\newenvironment{CSLReferences}[2] % #1 hanging-indent, #2 entry-spacing
 {\begin{list}{}{%
  \setlength{\itemindent}{0pt}
  \setlength{\leftmargin}{0pt}
  \setlength{\parsep}{0pt}
  % turn on hanging indent if param 1 is 1
  \ifodd #1
   \setlength{\leftmargin}{\cslhangindent}
   \setlength{\itemindent}{-1\cslhangindent}
  \fi
  % set entry spacing
  \setlength{\itemsep}{#2\baselineskip}}}
 {\end{list}}
\usepackage{calc}
\newcommand{\CSLBlock}[1]{\hfill\break\parbox[t]{\linewidth}{\strut\ignorespaces#1\strut}}
\newcommand{\CSLLeftMargin}[1]{\parbox[t]{\csllabelwidth}{\strut#1\strut}}
\newcommand{\CSLRightInline}[1]{\parbox[t]{\linewidth - \csllabelwidth}{\strut#1\strut}}
\newcommand{\CSLIndent}[1]{\hspace{\cslhangindent}#1}



\setlength{\emergencystretch}{3em} % prevent overfull lines

\providecommand{\tightlist}{%
  \setlength{\itemsep}{0pt}\setlength{\parskip}{0pt}}



 


\KOMAoption{captions}{tableheading}
\makeatletter
\@ifpackageloaded{bookmark}{}{\usepackage{bookmark}}
\makeatother
\makeatletter
\@ifpackageloaded{caption}{}{\usepackage{caption}}
\AtBeginDocument{%
\ifdefined\contentsname
  \renewcommand*\contentsname{Table of contents}
\else
  \newcommand\contentsname{Table of contents}
\fi
\ifdefined\listfigurename
  \renewcommand*\listfigurename{List of Figures}
\else
  \newcommand\listfigurename{List of Figures}
\fi
\ifdefined\listtablename
  \renewcommand*\listtablename{List of Tables}
\else
  \newcommand\listtablename{List of Tables}
\fi
\ifdefined\figurename
  \renewcommand*\figurename{Figure}
\else
  \newcommand\figurename{Figure}
\fi
\ifdefined\tablename
  \renewcommand*\tablename{Table}
\else
  \newcommand\tablename{Table}
\fi
}
\@ifpackageloaded{float}{}{\usepackage{float}}
\floatstyle{ruled}
\@ifundefined{c@chapter}{\newfloat{codelisting}{h}{lop}}{\newfloat{codelisting}{h}{lop}[chapter]}
\floatname{codelisting}{Listing}
\newcommand*\listoflistings{\listof{codelisting}{List of Listings}}
\makeatother
\makeatletter
\makeatother
\makeatletter
\@ifpackageloaded{caption}{}{\usepackage{caption}}
\@ifpackageloaded{subcaption}{}{\usepackage{subcaption}}
\makeatother
\usepackage{bookmark}
\IfFileExists{xurl.sty}{\usepackage{xurl}}{} % add URL line breaks if available
\urlstyle{same}
\hypersetup{
  pdftitle={my\_book\_again},
  pdfauthor={Kevin Williams},
  colorlinks=true,
  linkcolor={blue},
  filecolor={Maroon},
  citecolor={Blue},
  urlcolor={Blue},
  pdfcreator={LaTeX via pandoc}}


\title{my\_book\_again}
\author{Kevin Williams}
\date{2025-09-22}
\begin{document}
\maketitle

\renewcommand*\contentsname{Table of contents}
{
\hypersetup{linkcolor=}
\setcounter{tocdepth}{2}
\tableofcontents
}

\bookmarksetup{startatroot}

\chapter*{Preface}\label{preface}
\addcontentsline{toc}{chapter}{Preface}

\markboth{Preface}{Preface}

This is a Quarto book.

To learn more about Quarto books visit
\url{https://quarto.org/docs/books}.

\begin{Shaded}
\begin{Highlighting}[]
\DecValTok{1} \SpecialCharTok{+} \DecValTok{1}
\end{Highlighting}
\end{Shaded}

\begin{verbatim}
[1] 2
\end{verbatim}

\bookmarksetup{startatroot}

\chapter{Introduction}\label{introduction}

This is a book created from markdown and executable code.

See Knuth (1984) for additional discussion of literate programming.

\begin{Shaded}
\begin{Highlighting}[]
\DecValTok{1} \SpecialCharTok{+} \DecValTok{1}
\end{Highlighting}
\end{Shaded}

\begin{verbatim}
[1] 2
\end{verbatim}

\section{more}\label{more}

\subsection{more a}\label{more-a}

\subsection{more b}\label{more-b}

\subsection{more c}\label{more-c}

\section{less}\label{less}

\section{true}\label{true}

\subsection{ture z}\label{ture-z}

\subsubsection{true za}\label{true-za}

\subsection{ffg}\label{ffg}

\section{qwgwe=====}\label{qwgwe}

\part{Part 1}

\chapter{Meat}\label{meat}

\section{this is the meat of things}\label{this-is-the-meat-of-things}

\subsection{this is not the meat of
things}\label{this-is-not-the-meat-of-things}

\chapter{why\_me}\label{why_me}

\section{getting harder b bb}\label{getting-harder-b-bb}

\chapter{Hexwall}\label{hexwall}

\chapter{Projects - Hexwall}\label{projects---hexwall}

\section{Hex file resources}\label{hex-file-resources}

\subsection{Shiny App for creating
hex}\label{shiny-app-for-creating-hex}

\url{https://connect.thinkr.fr/hexmake/}

\chapter{FIX - get all of these
downloaded}\label{fix---get-all-of-these-downloaded}

https://github.com/rstudio/hex-stickers

\section{load hexwall.R function}\label{load-hexwall.r-function}

\begin{Shaded}
\begin{Highlighting}[]
\FunctionTok{source}\NormalTok{(}\StringTok{"\textasciitilde{}/Documents/r{-}studio{-}and{-}git/my\_book\_again/hexwall.R"}\NormalTok{)}
\end{Highlighting}
\end{Shaded}

\begin{verbatim}
Linking to ImageMagick 6.9.13.29
Enabled features: cairo, fontconfig, freetype, heic, lcms, pango, raw, rsvg, webp
Disabled features: fftw, ghostscript, x11
\end{verbatim}

\begin{Shaded}
\begin{Highlighting}[]
\NormalTok{test }\OtherTok{\textless{}{-}} \FunctionTok{hexwall}\NormalTok{(}\StringTok{"\textasciitilde{}/Documents/r{-}studio{-}and{-}git/my\_book\_again/my\_stickers"}\NormalTok{, }\AttributeTok{sticker\_row\_size =} \DecValTok{7}\NormalTok{, }\AttributeTok{sticker\_width =} \DecValTok{200}\NormalTok{)}
\end{Highlighting}
\end{Shaded}

\begin{verbatim}
Warning: `invoke()` was deprecated in purrr 1.0.0.
i Please use `exec()` instead.
\end{verbatim}

\begin{Shaded}
\begin{Highlighting}[]
\NormalTok{test}
\end{Highlighting}
\end{Shaded}

\includegraphics[width=4.67in,height=\textheight,keepaspectratio]{hexwall_files/figure-pdf/hexwall-1.png}

source(``\textasciitilde/Documents/r-studio-and-git/my\_hex\_stickers/hexwall/hexwall.R'')
\# call hexwall function and assign to ``test'' \# this sometimes get an
error -- try adjusting sticker\_row\_size test \textless-
hexwall(``\textasciitilde/Documents/r-studio-and-git/my\_hex\_stickers/my\_stickers'',
sticker\_row\_size = 7, sticker\_width = 200) test

png(``\textasciitilde/Documents/r-studio-and-git/my\_hex\_stickers/hexwall/samplehex/test123.png'')
hexwall(``\textasciitilde/Documents/r-studio-and-git/my\_hex\_stickers/hexwall/samplehex'',
sticker\_row\_size = 4, sticker\_width = 200) image\_write(test,
``\textasciitilde/Documents/r-studio-and-git/my\_hex\_stickers/hexwall/samplehex/test123.png'')
dev.off()

hex\_table \textless- data.table

\chapter{FIX - data.table is not working properly -- needs to be
fixed}\label{fix---data.table-is-not-working-properly-needs-to-be-fixed}

It seems to be reading, but not displaying in data.table format
correctly.\\
It works within R, but not in Quarto when rendered.

\chapter{FIX - data.table giving a warning about reading in last
line}\label{fix---data.table-giving-a-warning-about-reading-in-last-line}

\begin{Shaded}
\begin{Highlighting}[]
\CommentTok{\#library(data.table)}
\NormalTok{hex\_table }\OtherTok{\textless{}{-}} \FunctionTok{datatable}\NormalTok{(}\FunctionTok{read.csv}\NormalTok{(}\StringTok{"\textasciitilde{}/Documents/r{-}studio{-}and{-}git/my\_book\_again/my\_stickers\_data\_list/hex\_data.csv"}\NormalTok{))}
\end{Highlighting}
\end{Shaded}

\begin{verbatim}
Warning in read.table(file = file, header = header, sep = sep, quote = quote, :
incomplete final line found by readTableHeader on
'~/Documents/r-studio-and-git/my_book_again/my_stickers_data_list/hex_data.csv'
\end{verbatim}

\begin{Shaded}
\begin{Highlighting}[]
\NormalTok{hex\_table}
\end{Highlighting}
\end{Shaded}

\begin{verbatim}
PhantomJS not found. You can install it with webshot::install_phantomjs(). If it is installed, please make sure the phantomjs executable can be found via the PATH variable.
\end{verbatim}

\begin{verbatim}
file:////private/var/folders/sf/d810pt617h181j949xmh0yvh0000gn/T/Rtmp3DvtAd/file32ad1e97e20a/widget32ad5790e37.html screenshot completed
\end{verbatim}

\pandocbounded{\includegraphics[keepaspectratio]{hexwall_files/figure-pdf/unnamed-chunk-1-1.pdf}}

\begin{Shaded}
\begin{Highlighting}[]
\CommentTok{\#hex\_table\_as\_tibble \textless{}{-} as\_tibble(hex\_table)}
\CommentTok{\#hex\_table\_as\_tibble}
\end{Highlighting}
\end{Shaded}

\part{Projects}

\chapter{Projects - Creating
Projects2}\label{projects---creating-projects2}

\subsection{Creating a New Project}\label{creating-a-new-project}

To create a new project, follow these steps: Make this your working
directory: \textasciitilde/Documents/r-studio-and-git

In console: library(usethis)

usethis::create\_project(``\textasciitilde/Documents/r-studio-and-git/YOUR-PROJECT-NAME'')
This will create a new directory for your project and open it in
RStudio. It will also create a default R project file (.Rproj) for you.
It will also create a readme file for you. It will also create a
.gitignore file for you.

\subsection{Setting Up Version Control with
Git}\label{setting-up-version-control-with-git}

To set up version control with Git for your new project, follow these
steps: In console: usethis::use\_git() This will initialize a Git
repository in your project directory. It will also create a .git
directory to track changes. It will also make an initial commit with the
existing files in your project. \#\#\# Connecting to GitHub To connect
your local Git repository to a remote repository on GitHub, follow these
steps: First, create a new repository on GitHub without a README,
.gitignore, or license. Then, in console: usethis::use\_github() This
will create a new repository on GitHub and link it to your local Git
repository. It will also push your initial commit to the remote
repository on GitHub. \#\#\# Summary You have now created a new R
project, set up version control with Git, and connected it to a remote
repository on GitHub. You can now start working on your project and use
Git to track changes and collaborate with others.

\subsection{add starting files}\label{add-starting-files}

usethis::use\_r(``import'') \# for R scripts under ``R'' folder
usethis::use\_r(``tidy'') \# for R scripts under ``R'' folder

\subsection{add data folder}\label{add-data-folder}

usethis::use\_data\_raw() \# for raw data under ``data-raw''
usethis::use\_directory(``DATA'') \# for processed data under ``data''

\subsection{fixing and updating README.md file with github
info}\label{fixing-and-updating-readme.md-file-with-github-info}

When README.md file was first created, it does not render properly on
GitHub.\\
To fix this, follow these steps:\\
1. Open README.md file in RStudio.\\
2. Add the following YAML header at the top of the file:\\
--- title: ``Project Title'' \#optional author: ``Your Name'' \#
optional date: ``2024-06-10'' \#optional output: github\_document --- 3.
Save the file. 4. In console, run the following command to render the
README.md file: rmarkdown::render(``README.md'') 5. Commit and push the
changes to GitHub. This will update the README.md file to render
properly on GitHub with the specified title, author, and date.\\
You will need to render the README.md file each time you make changes to
it.

\chapter{Projects - Creating
Projects2}\label{projects---creating-projects2-1}

\subsection{Creating a New Project}\label{creating-a-new-project-1}

To create a new project, follow these steps: Make this your working
directory: \textasciitilde/Documents/r-studio-and-git

In console: library(usethis)

usethis::create\_project(``\textasciitilde/Documents/r-studio-and-git/YOUR-PROJECT-NAME'')
This will create a new directory for your project and open it in
RStudio. It will also create a default R project file (.Rproj) for you.
It will also create a readme file for you. It will also create a
.gitignore file for you.

\subsection{Setting Up Version Control with
Git}\label{setting-up-version-control-with-git-1}

To set up version control with Git for your new project, follow these
steps: In console: usethis::use\_git() This will initialize a Git
repository in your project directory. It will also create a .git
directory to track changes. It will also make an initial commit with the
existing files in your project. \#\#\# Connecting to GitHub To connect
your local Git repository to a remote repository on GitHub, follow these
steps: First, create a new repository on GitHub without a README,
.gitignore, or license. Then, in console: usethis::use\_github() This
will create a new repository on GitHub and link it to your local Git
repository. It will also push your initial commit to the remote
repository on GitHub. \#\#\# Summary You have now created a new R
project, set up version control with Git, and connected it to a remote
repository on GitHub. You can now start working on your project and use
Git to track changes and collaborate with others.

\subsection{add starting files}\label{add-starting-files-1}

usethis::use\_r(``import'') \# for R scripts under ``R'' folder
usethis::use\_r(``tidy'') \# for R scripts under ``R'' folder

\subsection{add data folder}\label{add-data-folder-1}

usethis::use\_data\_raw() \# for raw data under ``data-raw''
usethis::use\_directory(``DATA'') \# for processed data under ``data''

\subsection{fixing and updating README.md file with github
info}\label{fixing-and-updating-readme.md-file-with-github-info-1}

When README.md file was first created, it does not render properly on
GitHub.\\
To fix this, follow these steps:\\
1. Open README.md file in RStudio.\\
2. Add the following YAML header at the top of the file:\\
31b8e172-b470-440e-83d8-e6b185028602:dAB5AHAAZQA6AFoAUQBBAHgAQQBEAGcAQQBNAFEAQQA1AEEARABZAEEATQBBAEEAMQBBAEMAMABBAE0AQQBCAGgAQQBHAE0AQQBaAEEAQQB0AEEARABRAEEAWgBnAEIAaABBAEcAVQBBAEwAUQBBADQAQQBHAEkAQQBPAFEAQQA1AEEAQwAwAEEATwBRAEIAagBBAEQARQBBAFkAZwBBADMAQQBHAEkAQQBaAEEAQQAzAEEARwBNAEEATQBBAEIAbQBBAEQARQBBAAoAcABvAHMAaQB0AGkAbwBuADoATQBnAEEAdwBBAEQAVQBBAE0AUQBBAD0ACgBwAHIAZQBmAGkAeAA6AAoAcwBvAHUAcgBjAGUAOgBMAFEAQQB0AEEAQwAwAEEAQwBnAEIAMABBAEcAawBBAGQAQQBCAHMAQQBHAFUAQQBPAGcAQQBnAEEAQwBJAEEAVQBBAEIAeQBBAEcAOABBAGEAZwBCAGwAQQBHAE0AQQBkAEEAQQBnAEEARgBRAEEAYQBRAEIAMABBAEcAdwBBAFoAUQBBAGkAQQBDAEEAQQBJAHcAQgB2AEEASABBAEEAZABBAEIAcABBAEcAOABBAGIAZwBCAGgAQQBHAHcAQQBDAGcAQgBoAEEASABVAEEAZABBAEIAbwBBAEcAOABBAGMAZwBBADYAQQBDAEEAQQBJAGcAQgBaAEEARwA4AEEAZABRAEIAeQBBAEMAQQBBAFQAZwBCAGgAQQBHADAAQQBaAFEAQQBpAEEAQwBBAEEASQB3AEEAZwBBAEcAOABBAGMAQQBCADAAQQBHAGsAQQBiAHcAQgB1AEEARwBFAEEAYgBBAEEAZwBBAEEAbwBBAFoAQQBCAGgAQQBIAFEAQQBaAFEAQQA2AEEAQwBBAEEASQBnAEEAeQBBAEQAQQBBAE0AZwBBADAAQQBDADAAQQBNAEEAQQAyAEEAQwAwAEEATQBRAEEAdwBBAEMASQBBAEkAQQBBAGoAQQBHADgAQQBjAEEAQgAwAEEARwBrAEEAYgB3AEIAdQBBAEcARQBBAGIAQQBBAEsAQQBHADgAQQBkAFEAQgAwAEEASABBAEEAZABRAEIAMABBAEQAbwBBAEkAQQBCAG4AQQBHAGsAQQBkAEEAQgBvAEEASABVAEEAWQBnAEIAZgBBAEcAUQBBAGIAdwBCAGoAQQBIAFUAQQBiAFEAQgBsAEEARwA0AEEAZABBAEEASwBBAEMAMABBAEwAUQBBAHQAQQBBAD0APQAKAHMAdQBmAGYAaQB4ADoA:31b8e172-b470-440e-83d8-e6b185028602

\begin{enumerate}
\def\labelenumi{\arabic{enumi}.}
\setcounter{enumi}{2}
\tightlist
\item
  Save the file.
\item
  In console, run the following command to render the README.md file:
  rmarkdown::render(``README.md'')
\item
  Commit and push the changes to GitHub. This will update the README.md
  file to render properly on GitHub with the specified title, author,
  and date.\\
  You will need to render the README.md file each time you make changes
  to it.
\end{enumerate}

\chapter{Projects - Hex2}\label{projects---hex2}

\begin{itemize}
\tightlist
\item
  Get Hexwall working
\item
  Get Hexwall working with maps
\item
  Make a RRL Hex with Shiny App
\item
  Make a RRL Hex with hexSticker package
\item
  Get a Shiny Hexwall working
\item
  Get a Shiny Hexwall working with maps
\item
  Get a Shiny Hexwall working with multiple map choices
\item
  Get an updated Hexwall working with an input of a hex
\end{itemize}

\chapter{Projects - OpenAI}\label{projects---openai}

\section{use ellmer package}\label{use-ellmer-package}

\begin{Shaded}
\begin{Highlighting}[]
\FunctionTok{library}\NormalTok{(ellmer)}
\CommentTok{\#chat \textless{}{-} chat\_openai()}
\end{Highlighting}
\end{Shaded}

Sys.setenv(OPENAI\_API\_KEY = `\textless your key here')

\#\textgreater{} Using model = ``gpt-4.1''. chat\$chat('' What is the
difference between a tibble and a data frame? Answer with a bulleted
list ``)

chat\$chat(``Who created R?'', echo = FALSE)

\section{get prompt working with
openai}\label{get-prompt-working-with-openai}

\chapter{RRL.com Stuff}\label{rrl.com-stuff}

\subsection{As of 20251010}\label{as-of-20251010}

www.wordpress.com account to use is: User: kwill1992rrl which links to
kwill1992@hotmail.com There is another one (at least one other), which
is User: kwill1992 and links to kevin.williams@rrldataanalytics.com The
kwill1992 one does not have any websites linked to it. The kwill1992rrl
account has both the old one @ www.rrldataanalytics.com and the one in
development at the numbered IP address.

To get to the linked sites at www.wordpress.com, go to ``managed blogs''
button and this will come up: https://wordpress.com/sites

Only one is default site at any given time. You still need to login with
user account to edit each website. Each website has a ``user'' and
``kwill1992'' user. Different passwords for each for each website.

\part{Part 2}

\chapter{Concepts}\label{concepts}

\section{data.table vs DT}\label{data.table-vs-dt}

The \texttt{data.table} package is a powerful R package for data
manipulation and aggregation, known for its speed and efficiency with
large datasets. It provides an enhanced version of data frames with
additional features like fast grouping, indexing, and in-place
updates.\\
On the other hand, \texttt{DT} is an R package that provides an
interface to the JavaScript library DataTables. It is primarily used for
creating interactive tables in R Markdown documents and Shiny
applications. \texttt{DT} allows users to create sortable, searchable,
and paginated tables with ease, enhancing the user experience when
working with tabular data in web applications.

\subsubsection{for a quarto document and output as a website, use DT for
a data
table.}\label{for-a-quarto-document-and-output-as-a-website-use-dt-for-a-data-table.}

\section{Link to kaggle datasets to downloaded data directly into R
using the kaggle
API}\label{link-to-kaggle-datasets-to-downloaded-data-directly-into-r-using-the-kaggle-api}

library(RKaggle) \# for interacting with Kaggle API)\\
using RKaggle to download the dataset from Kaggle\\
Example:\\
superstore\_dataset \textless-
get\_dataset(``vivek468/superstore-dataset-final'')

github for RKaggle:\\
https://github.com/benyamindsmith/RKaggle

\section{Leaflet}\label{leaflet}

The \texttt{leaflet} package in R is a powerful tool for creating
interactive maps. It allows users to visualize spatial data with various
layers, markers, and pop-ups. The package is built on top of the Leaflet
JavaScript library, providing a user-friendly interface for R users to
create dynamic maps that can be embedded in R Markdown documents, Shiny
applications, or viewed in RStudio. With \texttt{leaflet}, users can
easily add tiles, polygons, and other geographical features to their
maps, making it a versatile choice for geospatial data visualization.

https://rstudio.github.io/leaflet/index.html\\
https://r-charts.com/spatial/interactive-maps-leaflet/\\
https://r-graph-gallery.com/package/leaflet.html\\
https://bookdown.org/nicohahn/making\_maps\_with\_r5/docs/leaflet.html
https://leafletjs.com/reference.html\\
https://www.geeksforgeeks.org/r-language/leaflet-package-in-r/\\
https://www.jla-data.net/eng/leaflet-in-r-tips-and-tricks/

\chapter{Summary}\label{summary}

In summary, this book has no content whatsoever.

\begin{Shaded}
\begin{Highlighting}[]
\DecValTok{1} \SpecialCharTok{+} \DecValTok{1}
\end{Highlighting}
\end{Shaded}

\begin{verbatim}
[1] 2
\end{verbatim}

\bookmarksetup{startatroot}

\chapter*{References}\label{references}
\addcontentsline{toc}{chapter}{References}

\markboth{References}{References}

\phantomsection\label{refs}
\begin{CSLReferences}{1}{0}
\bibitem[\citeproctext]{ref-knuth84}
Knuth, Donald E. 1984. {``Literate Programming.''} \emph{Comput. J.} 27
(2): 97--111. \url{https://doi.org/10.1093/comjnl/27.2.97}.

\end{CSLReferences}

\bookmarksetup{startatroot}

\chapter*{Errors}\label{errors}
\addcontentsline{toc}{chapter}{Errors}

\markboth{Errors}{Errors}

\subsection*{Github}\label{github}
\addcontentsline{toc}{subsection}{Github}

\begin{itemize}
\tightlist
\item
  to fix a git push commit holding up

  \begin{itemize}
  \tightlist
  \item
    rm .git/index.lock {[}in Terminal{]}
  \end{itemize}
\end{itemize}




\end{document}
